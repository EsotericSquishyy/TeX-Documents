\documentclass[11pt]{amsart}

\usepackage{amssymb}

\usepackage{amsmath}

\usepackage{amsthm}

\usepackage{mathrsfs}

\usepackage[pdftex]{graphicx}

\tolerance = 10000

\setlength{\oddsidemargin}{8mm}

\setlength{\evensidemargin}{8mm}
 
\setlength{\topmargin}{-5mm}

\setlength{\textwidth}{145mm}

\setlength{\textheight}{220mm}

\setlength{\parindent}{0pt}

\setlength{\parskip}{1.5ex plus 0.5ex minus 0.2ex}

\newtheorem{theorem}{Theorem}[section]

\newtheorem{lemma}[theorem]{Lemma}

\newtheorem{corollary}[theorem]{Corollary}

\newtheorem{proposition}[theorem]{Proposition}

\newtheorem{conjecture}[theorem]{Conjecture}

\theoremstyle{definition}

\newtheorem*{example}{Example}

\newtheorem*{definition}{Definition}

%\renewcommand\qedsymbol{$\blacksquare$}

\newcommand*{\lxor}{\veebar}



\begin{document}

\title{Homework 5}

\author{Jesse Cobb - 2PM Section}

\maketitle

\begin{itemize}

\item[3.2.9]
\begin{itemize}
    \item[d.] $\mathbb{Z}_7=\{\bar 0,\bar 1,\bar 2,\bar 3,\bar 4,\bar 5,\bar 6\}$
    
\end{itemize}

\item[3.2.10]
\begin{itemize}
    \item[c.] Congruent to $2(\text{mod }4)$ and congruent to $8(\text{mod }6)$.\\
              $14=2(\text{mod }4)$ and $14=8(\text{mod }6)$ \\
              $-10=2(\text{mod }4)$ and $-19=8(\text{mod }6)$
    
\end{itemize}

\item[3.4.1]
\begin{itemize}
    \item[a.] $6+6$ in $\mathbb{Z}_7$ \\
              $\bar 6+\bar 6=\bar{12}=\bar 5$

    \item[i.] $2^{25}$ in $\mathbb{Z}_7$ \\
              $\bar{2^{25}}=\bar{2^3}^8\bar2=\bar1^8\bar2=\bar 2$

    \item[j.] $5^{23}$ in $\mathbb{Z}_7$ \\
              $\bar{5^{23}}=\bar{25}^{11}\bar5=\bar{4^{11}}\bar5=\bar{2^{22}}\bar5
              =\bar{2^3}^7\bar{10}=\bar1^7\bar3=\bar3$

    \item[k.] $4^{44}$ in $\mathbb{Z}_7$ \\
              $\bar{4^{44}}=\bar{2^{88}}=\bar{2^3}^{29}\bar2=\bar1^{29}\bar2=\bar2$

    \item[l.] $2^{26}$ in $\mathbb{Z}_7$ \\
              $\bar{2^{25}}=\bar{2^3}^8\bar2^2=\bar1^8\bar2^2=\bar 4$
    
\end{itemize}

\item[3.4.7]
\begin{itemize}
    \item[a.] $238+496-44$ in $\mathbb{Z}_9$ \\
              $\bar{238}+\bar{496}-\bar{44}=\bar{690}=\bar{90}+\bar{450}+\bar{90}+\bar{45}+\bar{15}=\bar{15}=\bar6$
    
\end{itemize}

\item[4.1.1]
\begin{itemize}
    \item[c.] $R=\{(1,2),(2,1)\}$ \\
              $R$ is a function from $A$ to $B$ \\
              $A=\{1,2\},B=\mathbb{N}$ or $\mathbb{Z}$

    \item[e.] $R=\{(x,y)\in\mathbb{N}^2:x\le y\}$ \\
              $R$ is not a function on $\mathbb{N}$ \\
              Since $1\le 2$ and $1\le 1$ so $f(1)=1,2$

    \item[f.] $R=\{(x,y)\in\mathbb{Z}^2:y^2=x\}$ \\
              $R$ is not a function on $\mathbb{Z}$ \\
              Since $1=(-1)^2=(1)^2$ so $f(1)=-1,1$ \\
              and $x\ge 0$ since $y^2\ge 0$ so $\text{Dom}(R)\ne\mathbb{Z}$

    \item[i.] $R=\{(a,3),(b,2),(c,1)\}$ \\
              $R$ is not a function from $A$ to $B$ \\
              Since $A=\{a,b,c,d\}$ so $\text{Dom}(R)\ne A$
    
\end{itemize}

\item[4.1.2] $f:\mathbb{R}\to\mathbb{R}$ where $f(x)=\pm\sqrt{x}$ is not a function as $\mathbb{R}\ne\text{Dom}(R)$ since $\sqrt{x}$ is only defined when $x\ge 0$. Furthermore the function almost always has $2$ outputs for every input, an example being $f(1)=-1,1$ which deviates from the rules of a function.

\item[4.1.7] \begin{proof}
    Let $f:A\to B$ and $g:C\to D$ be functions. Now suppose $\text{Dom}(f)=\text{Dom}(g)$ and for all $x\in\text{Dom}(f),f(x)=g(x)$.
    \paragraph{($\subseteq$):} Let $(x,y)\in f$ this means that $f(x)=y$. Since $f(x)=g(x)\implies g(x)=y$ this means $(x,y)\in g$. Thus $f\subseteq g$.
    \paragraph{($\subseteq$):} Let $(x,y)\in g$ this means that $g(x)=y$. Since $f(x)=g(x)\implies f(x)=y$ this means $(x,y)\in f$. Thus $g\subseteq f$.\\
    Thus we've shown that $f=g$ by double containment, if $\text{Dom}(f)=\text{Dom}(g)$ and for all $x\in\text{Dom}(f),f(x)=g(x)$.
\end{proof}

\item[4.1.13] $f:\mathbb{Z}\to\mathbb{Z}_6$
\begin{itemize}
    \item[a.] $f(3)=\{\ldots-3,3,9,15\ldots\}=\{6k+3:k\in\mathbb{Z}\}$

    \item[b.] Image of $6=f(6)=f(0)=\bar0=\{6k:k\in\mathbb{Z}\}$

    \item[c.] A pre-image of $\bar 3=3$

    \item[d.] All pre-images of $\bar 1=6k+1$ where $k\in\mathbb{Z}$

\end{itemize}

\item[4.1.17] $\overline{\overline A}=m,\overline{\overline B}=n$
\begin{itemize}
    \item[a.] A function $f$ from $A$ to $B$ has $n^m$ possible forms.

    \item[b.] A function $f$ with only one element in the domain from $A$ to $B$ has $n$ possible forms.

\end{itemize}

\item[4.1.18]
\begin{itemize}
    \item[a.] \begin{proof}
        Let $f:A\to B$ where $xTy$ iff $f(x)=f(y)$ where $T$ is a relation on $A$.
        Let $x\in A$. Since $f$ is a function each value of $x$ maps to to a single value so $f(x)=f(x)$. Thus $xTx$ so $T$ is reflexive.
        Let $x,y\in A$. Assume $xTy$ so that $f(x)=f(y)$. This implies $f(y)=f(x)$ so $yTx$. Then $T$ is symmetric.
        Now let $x,y,z\in A$ and $xTy$ and $yTz$ so that $f(x)=f(y)$ and $f(y)=f(z)$. This implies $f(x)=f(z)$ so $xTz$. Then $T$ is transitive.
        Thus we've proved $T$ to be an equivalence relation on $A$.
    \end{proof}

    \item[b.] $f:\mathbb{R}\to\mathbb{R}$ is given by $f(x)=x^2$. \\
              $\bar0=\{0\}\qquad
              \bar2=\{4\}\qquad
              \bar4=\{16\}$

\end{itemize}

\item[4.2.1]
\begin{itemize}
	\item[g.] $f^{-1}$ exists only when $f$ is bijective \\
			  $f=\frac1{1-x}$ \\
			  if $x=1-\frac1y$ then $f(x)=\frac1{1-(1-\frac1y)}=y$ so $f$ is onto \\
			  $f:\mathbb{R}-\{1\}\to\mathbb{R}-\{0\}$ \\
			  Assume $x_1,x_2\in\mathbb{R}-\{0\}$ and $f(x_1)-f(x_2)$ then: \\
			  $\frac1{1-x_1}=\frac1{1-x_2}\implies 1-x_1=1-x_2\implies x_1=x_2$ \\
			  $f$ is bijective so $f^{-1}$ exists: $f^{-1}:\mathbb{R}-\{0\}\to\mathbb{R}-\{1\}$ \\
			  $f^{-1}(x)=1-\frac1x$

\end{itemize}

\item[4.2.2]
\begin{itemize}
    \item[b.] $f(x)=x^2+2x\qquad g(x)=2x+1$ \\
			  $\text{Dom}(f)=\mathbb{R}\quad\text{Rng}(f)=[-1,\infty)$ \\
			  $\text{Dom}(g)=\mathbb{R}\quad\text{Rng}(g)=\mathbb{R}$ \\
			  $(f\circ g)(x)=(2x+1)^2+2(2x+1)=4x^2+8x+3$ \\
			  $(g\circ f)(x)=2(x^2+2x)+1=2x^2+4x+1$

\end{itemize}

\item[4.2.3]
\begin{itemize}
    \item[b.] $f\circ g:\mathbb{R}\to[-1,\infty)$ \\
			  $g\circ f:\mathbb{R}\to[-1,\infty)$
    
\end{itemize}

\item[4.2.6] \begin{proof}
	Let $f:A\to B$ and $I_B:B\to B$ where $I_B(z)=z$. Now let $x\in A$ so that $f(x)=y$ where $y\in B$, so $I_B(y)=y$. This implies that $I_B(f(x))=f(x)$. Thus $I_B\circ f=f$ if $f:A\to B$ and $I_B:B\to B$.
\end{proof}

\item[4.2.7] \begin{proof}
		We can say that $\text{Dom}(f\circ f^{-1})=\text{Dom}(f^{-1})=\text{Rng}(f)$. This shows that $\text{Dom}(f\circ f^{-1})=C=\text{Dom}(I_C)$ since $I_C:C\to C$ is bijective by definition. Let $x\in C$ so thatt $I_C(x)=x$ and $f^{-1}(x)\in A$. So $f(f^{-1}(x))\in C$. So $f(f^{-1}(x))=x=I_C(x)$. Thus we've shown that $f\circ f^{-1}=I_C$ if $f$ and $f^{-1}$ are functions and $\text{Rng}(f)=C$.
\end{proof}

\item[4.3.1]
\begin{itemize}
	\item[b.] \begin{proof}
		Let $f:\mathbb{Z}\to\mathbb{Z}$ given by $f(x)=-x+1000$. Suppose $x,y\in\mathbb{Z}$ and let $x=(1000-y)\in\mathbb{Z}$ and note that:
		\begin{align*}
			f(x)
			&=f(1000-y)\\
			&=-(1000-y)+1000\\
			&=y-1000+1000\\
			&=y
		\end{align*}
		We've shown that for all $y\in\mathbb{Z}$ there exists an $x\in\mathbb{Z}$ such that $f(x)=y$. Thus $f$ is a surjection.
	\end{proof}

\item[d.] \begin{proof}
		Let $f:\mathbb{R}\to\mathbb{R}$ given by $f(x)=x^3$. Now suppose $x,y\in\mathbb{R}$ and let $x=\sqrt[3]{y}\in\mathbb{R}$ and note that:
		\begin{align*}
			f(x)
			&=f(\sqrt[3]{y})\\
			&=(\sqrt[3]{y})^3\\
			&=y
		\end{align*}
		We've shown that for any $y\in\mathbb{R}$ there exists an $x\in\mathbb{R}$ such that $f(x)=y$. Thus we've proved that $f$ is a surjection.
	\end{proof}

\item[e.] \begin{proof}
		We'll show that $f:\mathbb{R}\to\mathbb{R}$, given by $f(x)=\sqrt{x^2+5}$, is not a surjection. We'll show this by contradiction and state that $f(x)=-1$. Note that:
		\begin{align*}
			-1=\sqrt{x^2+5}
			&\implies (-1)^2=x^2+5 \\
			&\implies 1-5=x^2 \\
			&\implies \sqrt{-4}=x
		\end{align*}
		We've shown that $f(x)=-1$ is a contradiction as $x=\sqrt{-4}\notin\mathbb{R}$ and $-1\in\mathbb{R}$. Thus we've shown that $f$ is not a surjection by showing a value that is not mapped onto that is in the codomain.
	\end{proof}

\item[h.] \begin{proof}
		Let $f:\mathbb{R}^2\to\mathbb{R}$, given by $f(x,y)=x-y$, is a surjection. Let $x\in\mathbb{R}$ and $y=0\in\mathbb{R}$ so $(x,y)=(x,0)$. Then $f(x,y)=x-0=x\in\mathbb{R}$. Thus we've proved that $f$ is a surjection.
	\end{proof}

\end{itemize}

\item[4.3.2]
\begin{itemize}
	\item[b.] \begin{proof}
		Let $x_1,x_2\in\mathbb{Z}$ and assume for $f:\mathbb{Z}\to\mathbb{Z}$, given by $f(x)=-x+1000$, that $f(x_1)=f(x_2)$. Note that:
		\begin{align*}
			f(x_1)=f(x_2)
			&\implies -x_1=1000=-x_2+1000\\
			&\implies -x_1=-x_2\\
			&\implies x_1=x_2
		\end{align*}
		We've shown that if $f(x_1)=f(x_2)$ for $x_1,x_2\in\mathbb{Z}$ then $x_1=x_2$. Thus we've shown that $f$ is an injection.
	\end{proof}

	\item[d.] \begin{proof}
		Let $x_1,x_2\in\mathbb{Z}$ and assume for $f:\mathbb{R}\to\mathbb{R}$, given by $f(x)=x^3$, that $f(x_1)=f(x_2)$. Note that:
		\begin{align*}
			f(x_1)=f(x_2)
			&\implies x_1^3=x_2^3\\
			&\implies x_1=x_2
		\end{align*}
		We've shown that if $f(x_1)=f(x_2)$ for $x_1,x_2\in\mathbb{R}$ then $x_1=x_2$. Thus we've shown that $f$ is an injection.
	\end{proof}

	\item[e.] \begin{proof}
		To show that $f:\mathbb{R}\to\mathbb{R}$, given by $f(x)=\sqrt{x^2+5}$, is not an injection we simply show the case of $f(x)=\sqrt{6}$. Since $f(-1)=f(1)=\sqrt{6}$ where $-1,1\in\mathbb{R}$, we've proved that $f$ is not an injection.	
	\end{proof}

	\item[h.] \begin{proof}
		To show that $f:\mathbb{R}^2\to\mathbb{R}$, given by $f(x,y)=x-y$ is not an injection we simply show the case $f(x,y)=0$. Since $f(1,-1)=f(2,-2)=0$ where $(1,-1),(2,-2)\in\mathbb{R}^2$ but $(1,-1)\ne(2,-2)$ we have proved that $f$ is not an injection.
	\end{proof}

\end{itemize}

\item[4.3.5] \begin{proof}
		Let $f:A\xrightarrow{\text{onto}} B$ and $g:B\xrightarrow{\text{onto}} C$. Then $\text{Rng}(f)=B$ and $\text{Dom}(g)=B$. Let $b\in B$ so that $f(a)=b$ for some $a\in A$ and $g(b)=c\in C$. Then $g(f(a))=f(b)=c$ so we've proved that $g\circ f$ is a surjection.
\end{proof}

\item[4.3.6] \begin{proof}
	Let $f:A\to B$, $g:B\to C$, and $g\circ f:A\xrightarrow{1-1} C$ be functions. Now let $x_1,x_2\in A$ and let $f(x_1)=f(x_2)$. Note that:
	\begin{align*}
		f(x_1)=f(x_2)
		&\implies g(f(x_1))=g(f(x_2)) \\
		&\implies (g\circ f)(x_1)=(g\circ f)(x_2)\\
		&\implies x_1=x_2
	\end{align*}
	We've shown that if $f(x_1)=f(x_2)$ then $x_1=x_2$. Thus we've proved that if $g\circ f$ is an injection then $f$ must also be an injection.
\end{proof}

\item[4.3.7] \begin{proof}
	Let $f:A\xrightarrow{1-1} B$ be a function. Assume there exists a restriction of $f$, $f|_D$, and $x_1,x_2\in D$. Since $D\subseteq A$, by the definition of a restriction, then $x_1,x_2\in A$. Then $f(x_1)=f(x_2)\implies x_1=x_2$. Since this is true for any $x_1$ and $x_2$ in $D$ then $f|_D$ is an injection. Thus we've shown any restriction of an injection is, itself, an injection. 
\end{proof}

\item[4.3.8] \begin{proof}
		Let $h:A\xrightarrow{\text{onto}}C$, $g:B\xrightarrow{\text{onto}} D$, $A\cap B=\varnothing$ and $C\cap D=\varnothing$. Then $\text{Dom}(h\cup g)=A\cup B$ and $\text{Rng}(h\cup g)=C\cup D$. \\
		Let $y\in C$ then $(h\cup g)(x)=h(x)$ and $h(x)$ is a surjection so for all $y$ there exists an $x\in A$ so that $(h\cup g)(x)=y$. \\
		Let $y\in D$ then $(h\cup g)(x)=g(x)$ and $g(x)$ is a surjection so for all $y$ there exists an $x\in B$ so that $(h\cup g)(x)=y$. \\
		Thus we've shown that in every possible case of $y\in C\cup D$ there exists an $x\in A\cup B$ so that $(h\cup g)(x)=y$, thus we've proved that $h\cup g$ is an injection.
\end{proof}

\end{itemize}

\end{document}
