\documentclass[11pt]{amsart}

\usepackage{amssymb}

\usepackage{amsmath}

\usepackage{amsthm}

\usepackage{mathrsfs}

\usepackage[pdftex]{graphicx}

\tolerance = 10000

\setlength{\oddsidemargin}{8mm}

\setlength{\evensidemargin}{8mm}

\setlength{\topmargin}{-5mm}

\setlength{\textwidth}{145mm}

\setlength{\textheight}{220mm}

\setlength{\parindent}{0pt}

\setlength{\parskip}{1.5ex plus 0.5ex minus 0.2ex}

\newtheorem{theorem}{Theorem}[section]

\newtheorem{lemma}[theorem]{Lemma}

\newtheorem{corollary}[theorem]{Corollary}

\newtheorem{proposition}[theorem]{Proposition}

\newtheorem{conjecture}[theorem]{Conjecture}

\theoremstyle{definition}

\newtheorem*{example}{Example}

\newtheorem*{definition}{Definition}

%\renewcommand\qedsymbol{$\blacksquare$}

\newcommand*{\lxor}{\veebar}



\begin{document}

\title{Homework 6}

\author{Jesse Cobb - 3PM Section (Mon,Wed)}

\maketitle

\begin{itemize}

\item[1.] $m$ deductible, $X$ denotes true cost of damages, and $Y$ denote amount of money actually paid by you. $X\sim\text{Exp}(\lambda)$ where $\lambda > 0$. 
\begin{itemize}
    \item[a.] $Y=g(X)=\begin{cases}
        X &0\le Y\le m\\
        m &m<Y
    \end{cases}$

    \item[b.] $E[Y]=E[g(X)]=\int_0^m\lambda xe^{-\lambda x}dx+\int_m^\infty \lambda e^{-\lambda x}dx\\
              =[-xe^{-\lambda x}-\frac{e^{-\lambda x}}{\lambda}]_0^m+[-e^{-\lambda x}]_m^\infty \\
              =-me^{-\lambda m}-\frac{e^{-\lambda m}}{\lambda}+\frac1\lambda+e^{-\lambda m} \\
              =e^{-\lambda m}(-m-\frac{1}{\lambda}+1)+\frac1\lambda$

    \item[c.] $F_Y(y)=\begin{cases}
        1-e^{-\lambda y} &0\le y\le m\\
        1-e^{-\lambda m}&m<y \\
        0 &\text{otherwise}
    \end{cases}$

    \item[d.] $Y$ is continuous in the sense that it is measured via density functions and that there are an uncountably infinite number of valid random variables $Y$. Though it does exhibit some discrete tendencies such as grouping of values. Therefore $Y$ is neither.
    
\end{itemize}

\item[2.] Triangle: $(0,0),(1,0),(0,1)$. $(X,Y)$ are uniformly distributed.
\begin{itemize}
    \item[a.] $f_{X,Y}(x,y)=\frac{1}{\frac 12}=2$ \\
              $f_X(x)=\int_0^{1-x}2dy=2y]_0^{1-x}=2-2x$ \\
              $f_Y(y)=\int_0^{1-y}2dx=2x]_0^{1-y}=2-2y$

    \item[b.] $E[X]=\int_0^1x(2-2x)dx=\int_0^12x-2x^2dx=x^2-\frac{2x^3}3]_0^1=\frac 13$ \\
              $E[Y]=\int_0^1y(2-2y)dy=\int_0^12y-2y^2dy=y^2-\frac{2y^3}3]_0^1=\frac 13$

    \item[c.] $\int_0^1\int_0^{1-x}2xydydx=\int_0^1xy^2]_0^1dx=\int_0^1xdx=1$

    \item[d.] $P(X>Y)=\frac 12$, since $y=x$ line cuts region in half.
    
\end{itemize}

\item[3.]$f_{X,Y}(x,y)=\begin{cases}
    c\cdot (\frac yx)^4 &(x,y)\in \mathcal{R}\\
    0 &\text{otherwise}
\end{cases}$ \\
$\mathcal{R}$ is the region in first quadrant under $y=\min\{x,1\}$ and $c>0$.
\begin{itemize}
    \item[a.] $\int_0^1\int_0^xc\cdot (\frac yx)^4dydx+\int_1^\infty\int_0^1c\cdot (\frac yx)^4dydx \\
              =\frac c5\int_0^1\frac{y^5}{x^4}]_0^xdx+\frac c5\int_1^\infty\frac{y^5}{x^4}]_0^1dx \\
              =\frac c5\int_0^1xdx+\frac c5\int_1^\infty\frac1{x^4}dx\\
              =\frac c{10}-\frac c{15}[x^{-3}]_1^\infty \\
              =\frac c{10}+\frac c{15}=\frac{5c}{30}=\frac c6=1\implies c=6$

    \item[b.] $P(X+Y\ge 2)=6\int_1^2\int_{2-x}^1(\frac yx)^4dydx+6\int_2^\infty\int_0^1(\frac yx)^4dydx$

    \item[c.] $f_X(x)=6\int_0^x(\frac yx)^4dy+6\int_0^1(\frac yx)^4dy
              =\frac {6x}{5}+\frac{6}{5x^4}$

    \item[d.] $f_Y(y)=6\int_y^1(\frac yx)^4dx+6\int_1^\infty(\frac yx)^4dx
              =-2\frac {y^4}{x^3}]_y^1-2\frac {y^4}{x^3}]_1^\infty\\
              =-2[y^4-y]-2[-y^4]=-2y^4+2y+2y^4=2y$

    \item[e.] $E[X]=6\int_0^1\int_0^x\frac {y^4}{x^3}dydx+6\int_1^\infty\int_0^1\frac {y^4}{x^3}dydx
              =\frac65\int_0^1\frac {y^5}{x^3}]_0^xdx+\frac65\int_1^\infty\frac {y^5}{x^3}]_0^1dx\\
              =\frac65\int_0^1x^2dx+\frac65\int_1^\infty\frac 1{x^3}dx
              =\frac25[x^3]_0^1-\frac35[\frac 1{x^2}]_1^\infty\\
              =\frac25+1=\frac{7}5$

    \item[f.] $E[Y]=\int_0^12y^2dy=\frac{2y^3}3]_0^1=\frac23$
    
\end{itemize}

\item[4.] Let $Y-X^\beta$ where $X\sim\text{Exp}(1)$ and $\beta =3$.
\begin{itemize}
    \item[a.] $P(Y>s+t|Y>s)=\frac{P(Y>s+t)}{P(Y>s)}$ \\
              $P(Y>s+t)=P(X>\sqrt[3]{s+t})=1-F_X(\sqrt[3]{s+t})=e^{-\sqrt[3]{s+t}}$ \\
              $P(Y>s)=P(X>\sqrt[3]{s})=1-F_X(\sqrt[3]{s})=e^{-\sqrt[3]{s}}$ \\
              $P(Y>t)=P(X>\sqrt[3]{t})=1-F_X(\sqrt[3]{t})=e^{-\sqrt[3]{t}}$ \\
              $P(Y>s+t|Y>s)=e^{-\sqrt[3]{s+t}+\sqrt[3]{s}}\ne P(Y>t)$ \\
              Not memoryless

    \item[b.] $E[Y]=E[X^3]=\int_0^\infty x^3e^{-x}dx=6$ \\
              $E[X^6]=\int_0^\infty x^6e^{-x}dx=720$ \\
              $\text{Var}(Y)=\text{Var}(X^3)=E[X^6]-E[X^3]^2=720-36=684$
    
\end{itemize}

\item[5.] $p_{X,Y}(x,y)=\begin{cases}
    (y-1)(\frac 12)^{x+y} &x\in\{1,2\ldots\},y\in\{2,3\ldots\} \\
    0 &\text{otherwise}
\end{cases}$
\begin{itemize}
    \item[a.] $\displaystyle
              \sum_{y=2}^\infty\sum_{x=1}^\infty (y-1)(\frac 12)^{x+y}=1$  Valid PMF

    \item[b.] $\displaystyle
              p_X(x)\sum_{y=2}^\infty (y-1)(\frac 12)^{x+y}=2^{-x} \\
              p_Y(y)\sum_{x=1}^\infty (y-1)(\frac 12)^{x+y}=2^{-y}(y-1)$

    \item[c.] $\displaystyle
              E[X]=\sum_{y=2}^\infty\sum_{x=1}^\infty x(y-1)(\frac 12)^{x+y}=2\\
              E[Y]=\sum_{y=2}^\infty\sum_{x=1}^\infty y(y-1)(\frac 12)^{x+y}=4$

    \item[d.] $\displaystyle
              E[XY]=\sum_{y=2}^\infty\sum_{x=1}^\infty xy(y-1)(\frac 12)^{x+y}=8 \\
              E[XY]=E[X]E[Y]=8$
    
\end{itemize}

\item[6.]$f(x,y)=\begin{cases}
    c(1-y) &0<x<y<1 \\
    0 &\text{otherwise}
\end{cases}$
\begin{itemize}
    \item[a.] $c\int_0^1\int_0^y(1-y)dxdy=c\int_0^1y-y^2dy=c[\frac12-\frac 13]=\frac c6=1\implies c=6$

    \item[b.] $P(X<\frac 12 | Y>\frac 23)=\frac{P(X<\frac 12 \cap Y>\frac 23)}{P(Y>\frac 23)}$ \\
              $P(X<\frac 12 \cap Y>\frac 23)
              =6\int_\frac23^1\int_0^y(1-y)dxdy-6\int_\frac23^1\int_\frac12^y(1-y)dxdy \\
              =6\int_\frac23^1(y-y^2)dy-6\int_\frac23^1\frac{3y}2-y^2-\frac12dy \\
              =6[\frac{y^2}2-\frac{y^3}3]_\frac23^1-6[\frac{3y^2}4-\frac{y^3}3-\frac y2]_\frac23^1 \\
              =\frac7{27}-\frac1{27}=\frac6{27}=\frac29$ \\
              $P(Y>\frac23)=6\int_\frac23^1\int_0^y(1-y)dxdy=\frac7{27}$ \\
              $P(X<\frac 12 | Y>\frac 23)=\frac67$

    \item[c.] $E[X]=6\int_0^1\int_0^yx(1-y)dxdy=\frac 14$

    \item[d.] $E[Y]=6\int_0^1\int_0^yy(1-y)dxdy=\frac 12$
    
\end{itemize}

\item[7.] $f(x,y)=\begin{cases}
    \frac{12}7(xy+y^2) &0\le x\le 1\text{ and }0\le y\le 1\\
    0 &\text{otherwise}
\end{cases}$
\begin{itemize}
    \item[a.] $P(X<2Y)=\int_0^1\int_\frac x2^1\frac{12}7(xy+y^2)dydx=\frac{13}{14}$

    \item[b.] $f(x)=\int_0^1\frac{12}7(xy+y^2)dy=\frac{12}7(\frac13+\frac x2)$ \\
              $f(y)=\int_0^1\frac{12}7(xy+y^2)dx=\frac{12}7(y^2+\frac y2)$

    \item[c.] $E[X^2Y]=\int_0^1\int_0^1\frac{12}7x^2y(xy+y^2)dydx=\frac 27$

\end{itemize}


\end{itemize}

\end{document}