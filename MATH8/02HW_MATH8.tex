\documentclass[11pt]{amsart}

\usepackage{amssymb}

\usepackage{amsmath}

\usepackage{amsthm}

\usepackage[pdftex]{graphicx}

\tolerance = 10000

\setlength{\oddsidemargin}{8mm}

\setlength{\evensidemargin}{8mm}

\setlength{\topmargin}{-5mm}

\setlength{\textwidth}{145mm}

\setlength{\textheight}{220mm}

\setlength{\parindent}{0pt}

\setlength{\parskip}{1.5ex plus 0.5ex minus 0.2ex}

\newtheorem{theorem}{Theorem}[section]

\newtheorem{lemma}[theorem]{Lemma}

\newtheorem{corollary}[theorem]{Corollary}

\newtheorem{proposition}[theorem]{Proposition}

\newtheorem{conjecture}[theorem]{Conjecture}

\theoremstyle{definition}

\newtheorem*{example}{Example}

\newtheorem*{definition}{Definition}

%\renewcommand\qedsymbol{$\blacksquare$}

\newcommand*{\lxor}{\veebar}



\begin{document}

\title{Homework 2}

\author{Jesse Cobb - 2PM Section}

\maketitle

\begin{itemize}

\item[1.5.3]
\begin{itemize}
    \item[c.] \begin{proof}
        Solving by the contrapositive, let $x$ be even, so that $x=2k$ for some $k\in\mathbb{Z}$. We will prove that if this assumption is correct $4\mid x^2$. Since $x=2k$, $x^2=(2k)^2=4k^2=2h$ where $h=k^2$ is an integer. Thus we've proved by the contrapositive that if $4\nmid x^2$ then $x$ is odd.
    \end{proof}

    \item[d.] \begin{proof}
        Solving by the contrapositive, let both $x$ and $y$ be odd, so that $x=2k+1$ and $y=2j+1$ for some $k,j\in\mathbb{Z}$. We'll prove that if this assumption is correct $xy$ is odd. Since $x=2k+1$ and $y=2j+1$, then $xy=(2k+1)(2j+1)=4kj+2j+2k+1=2(2kj+j+k)+1=2h+1$ where $h=2kj+j+k$ is an integer. Thus we've proved through the contrapositive that if $x+y$ is even then $x$ or $y$ have to be even.
    \end{proof}

    \item[e.] \begin{proof}
        Solving by the contrapositive, let $x$ and $y$ have different parity then, without loss of generality, assume $x$ is even and $y$ is odd so that $x=2k$ and $y=2j+1$ for some $k,j\in\mathbb{Z}$. So that $x+y=2k+2j+1=2(k+j)+1=2h+1$ where $h=k+j$ is an integer. Thus we've proved through the contrapositive that if $x+y$ is even then $x$ and $y$ must be the same parity.
    \end{proof}

    \item[f.] \begin{proof}
        Solving by the contrapositive, let at least $x$ or $y$ be odd. Without loss of generality consider the following cases:
        \paragraph{Case 1:}
            If $x$ is odd and $y$ is even, such that $x=2k+1$ and $y=2j$ for $k,j\in\mathbb{Z}$ then $xy=(2k+1)2j=4kj+2j=2(2kj+j)=2h$ where $h=2kj+j$ is an integer.
        \paragraph{Case 2:}
            If $x$ and $y$ are both even, such that $x=2k$ and $y=2j$ for $k,j\in\mathbb{Z}$ then $xy=(2k)2j=4kj=2(2kj)=2h$ where $h=2kj$ is an integer. \\
        Thus we've proved by the contrapositive that if $xy$ is odd then $x$ and $y$ must both be odd.
    \end{proof}

    \item[g.] \begin{proof}
        Solving by the contrapositive, let $x$ be odd, so that $x=2k+1$ for some $k\in\mathbb{Z}$. We'll show that $8\mid x^2-1$ if the assumption is true. Since $x=2k+1$ then $x^2-1=(2k+1)^2-1=4k^2+4k+1-1=4(k^2+k)=4k(k+1)$. Since $a(a+1)$ is even for any integer $4k(k+1)=4*2(h)=8h$ where $h\in\mathbb{Z}$ therefore $x^2-1=8h$ if $x$ is odd. Thus we've proved through the contrapositive that if $8\nmid x^2-1$ then $x$ is even.
    \end{proof}

    \item[h.] \begin{proof}
        Solving by the contrapositive, let $x\mid z$ so that $z=xk$ for some $k\in\mathbb{Z}$. Then $yz=y(xk)=x(yk)=xj$ where $j=yk$ is an integer. Thus we've proved by contrapositive that if $x\nmid yz$ then $x\nmid z$.
    \end{proof}

\end{itemize}

\item[1.5.4]
\begin{itemize}
    \item[d.] \begin{proof}
        Solving by the contrapositive, let the real number $x$ be greater than or equal to $1$, $x\ge 1$. Therefore $x-1\ge 0$ which implies $x^2-x\ge 0$ since $x$ is always positive. This is equivalent to $(x+1)(x-1)\ge 0$. Thus we've proved through the contrapositive that if $(x+1)(x-1)< 0$ then $x<1$.
    \end{proof}

    \item[e.] \begin{proof}
        Solving by the contrapositive, let the real number $x$ then assume $3\ge x\ge 1$. Since $3\ge x$ then $0\ge x-3$ and since $1\le x$ then $0\le x-1$. With these equivalence statement we can conclude the $0\ge (x-3)(x-1)\implies 0\ge x^2-4x+3\implies -3\ge x^2-4x\implies -3\ge x(x-4)$. Thus we've proved through contrapositive that if $x(x-4)>0$ then $x<1$ or $x>3$.
    \end{proof}

\end{itemize}

\item[1.5.6]
\begin{itemize}
    \item[d.] \begin{proof}
        Let $a-b$ be odd such that $a-b=2k+1$ for $k\in\mathbb{Z}$. Seeking a contradiction, assume $a+b$ is even such that $a+b=2j$ for some integer $j$. Therefore $a-b=2k+1\implies a-b+2b=2k+1\implies 2j+2b=2k+1\implies 2(j+b)=2k+1\implies 2h=2k+1$ where $h=j+b$ is an integer. This results in a contradiction showing that an even number ($2h$) is equal to an odd number ($2k+1$). Therefore the claim follows that if $a-b$ is odd then $a+b$ must be odd.
    \end{proof}

    \item[e.] \begin{proof}
        Let $a<b$ and $ab<3$ for $a,b\in\mathbb{Z}$ where $a,b>0$. Seeking a contradiction, assume $a\ne 1$. Since $ab <3$ then $0<b<3$ which implies $0<a<2$. This means the only valid value for $a$ is $1$ which contradicts our assumption. Thus we've proved the claim if $a<b$ and $ab<3$ then $a=1$.
    \end{proof}

\end{itemize}

\item[1.5.7]
\begin{itemize}
    \item[a.] \begin{proof}
        Assume $a$ and $b$ are positive integers. Therefore $a\mid b$ if and only if:
        \begin{align*}
            b=ak\text{ where }k\in\mathbb{Z} &\iff \\
            bc=ack &\iff \\
            ac\mid bc
        \end{align*}
        Since the final statement is true all the previous statements are true including $a\mid b$. Thus we've proved the claim that $ac\mid bc$ if and only if $a\mid b$.
    \end{proof}

    \item[b.] \begin{proof}
        Let $a$ and $b$ be positive integers. We'll prove that $a+1\mid b$ and $b\mid b+3$ if and only if $a=2$ and $b=3$. \\
        ($\implies$) Assume that $a+1\mid b$ and $b\mid b+3$ so that $b=(a+1)k$ and $b+3=bj$ for some $k,j\in\mathbb{Z}$. Then $(a+1)k+3=(a+1)kj\implies ak+k+3=akj+kj\implies 3=ak(j-1)+k(j-1)\implies 3=h(a+1)\implies a+1\mid 3$. This requires $a+1$ to be $1$ or $3$, where $1$ is impossible since it would require $a=0$, therefore $a=2$. Thus $b=3k$ which means $3\mid b$ and $3=b(j-1)=bh$ where $b\mid 3$ which means that $b=3$. \\
        ($\impliedby$) Now assume, for the reverse direction, that $a=2$ and $b=3$. This means $a+1=3\implies a+1\mid 3$ which means $a+1\mid b$ and $b\mid b+3$ since $3\mid 6$. \\
        Thus we've proved that $a+1\mid b$ and $b\mid b+3$ if and only if $a=2$ and $b=3$ by showing their implications in both directions are true.
    \end{proof}

\end{itemize}

\item[1.5.9] \begin{proof}
        Seeking a contradiction, for the natural number $n$ assume $\frac n{n+1}\le \frac n{n+2}$. Since $n\ge 1$ we can say that $\frac 1{n+1}\le \frac 1{n+2}\implies n+2\le n+1\implies 2\le 1$ which is a contradiction. Thus the claim stands that for any natural number $n$ $\frac n{n+1}> \frac n{n+2}$.
    \end{proof}

\item[1.5.10] \begin{proof}
        Seeking a contradiction, assume that $\sqrt{5}$ is rational such that $\sqrt{5}=\frac ab$ where $a,b\in\mathbb{Z}$. Then $5=\frac {a^2}{b^2}\implies a^2=5b^2$. By the fundamental theorem of arithmetic $a$ should have $k$ factors of the prime factor $5$ where $k\in\mathbb{Z}$. Thus $a^2$ should have $2k$ prime factors of $5$. Similarly $b^2$ should have $2j$ prime factors of $5$ where $j\in\mathbb{Z}$. Therefore we have a contradiction as $a^2$ has $2k$ factors of $5$ and $5b^2$ has $2j+1$ factors of $5$. Thus the claim follows that $\sqrt{5}$ is irrational.
    \end{proof}

\item[1.5.11] \begin{proof}
        Assume for the real numbers $x$, $y$, and $z$ between $0$ and $1$ that $0<x<y<z<1$. Seeking a contradiction we'll assume that the distance between each number, $x$, $y$, and $z$, is a space greater than or equal to $\frac 12$. So we say that $a=x-0$ is the distance from $0$ to $x$, $b=y-x$ is the distance from $x$ to $y$, $c=z-y$ is the distance from $y$ to $z$, where $a,b,c\in\mathbb{R}$ and $a+b+c<1$. If we set $x\approx 0$ then $a\approx 0$ leaving us with $b+c<1$. However we get a contradiction when stating that $b\ge \frac 12$ and $c\ge \frac 12$, as at their lowest values we still end with $\frac 12+\frac 12<1$ which is false. Thus by contradiction we can conclude that at least two of the real numbers, $x$, $y$, and $z$, between $0$ and $1$ must be within $\frac 12$ units of one another.
    \end{proof}

\item[1.5.12]
\begin{itemize}
    \item[a.] C, the proof is all over the place. This is neither a proof by contrapositive (if $m$ is even $m^2$ is even), not a proof by contradiction (show contradiction of: if $m^2$ is even assume $m$ is odd), and also not a direct proof as it assumes the opposite of what the proof asks for ($m^2$ is odd). It is partially correct as the claim is correct by the following proof:
    \begin{proof}
        Solving by the contrapositive, assume the integer $m$ is even such that $m=2k$ for some $k\in\mathbb{Z}$. Then $m^2=(2k)^2=2*2k^2=2j$ where $j=2k^2$ is an integer. Thus by proving the contrapositive we can say that if $m^2$ is odd $m$ is odd.
    \end{proof}
    
\end{itemize}

\item[1.6.1]
\begin{itemize}
    \item[b.] \begin{proof}
        Since $15m+12n=3\implies 5m+4n=1\implies 5m=1-4n$. Assume $m=1$ and $n=-1$, then $15m+12n+3$ turns into $15-12=3$ which is true. Thus we've proved that there exists integers $m$ and $n$ that satisfy $15m+12n=3$.
    \end{proof}

    \item[d.] \begin{proof}
        Since $15m+12n=1\implies 3(5m+4n)=1\implies 3k=1$ where $k=5m+4n$ is an integer. This is a contradiction as $3\nmid 1$. Thus we've proved that there exists no integers $m$, $n$ that satisfy $15m+12n=1$.
    \end{proof}

\end{itemize}

\item[1.6.4]
\begin{itemize}
    \item[a.] \begin{proof}
        Suppose $x=41$ then $x^2+x+41=41^2+41+41=43(41)$. This disproves the statement that for all integers $x$, $x^2+x+41$ is prime as $43(41)$ clearly has at least one other factor besides itself and $1$.
    \end{proof}

    \item[b.] \begin{proof}
        For all real numbers $x$ there exists a real number $y$ that satisfies the statement $x+y=0$. Since $x+y=0\implies y=-x\in\mathbb{R}$. Thus we've proved that there always exists at least one real number $y$ that satisfies $x+y=0$.
    \end{proof}

    \item[c.] \begin{proof}
        Suppose the real numbers $x=2$ and $y=1$. Then $y^x>x\implies 1>2$ which is clearly a contradiction. Thus we've disproved that for all real numbers $x>1$ and $y>0$, $y^x>x$.
    \end{proof}

    \item[d.] \begin{proof}
        Suppose the integers $a=4$, $b=2$, and $c=2$. Then this fulfills the assumption that $a\mid bc$, as $2\cdot 4k$ for some $k\in\mathbb{Z}$ where $k=1$. However this fails both $a\mid b$ and $a\mid c$ as $2\ne 4j$ and $2\ne 4l$ for some $j,l\in\mathbb{Z}$. Thus we have disproved the statement that if an integer $a$ divides the product of two integers $b$ and $c$ then $a$ must also divide one of the other integers $b$ or $c$.
    \end{proof}
    
\end{itemize}

\item[1.6.6]
\begin{itemize}
    \item[a.] \begin{proof}
        The natural number $n$ is greater than or equal to $1$ by definition of a natural number. Since $n\ge 1\implies 1\ge \frac 1n$. Thus we've proved for all natural numbers $n$, $\frac 1n\le 1$.
    \end{proof}

    \item[b.] \begin{proof}
       Assume there exists a natural number $n$ that $\frac 1n <.13$. Based on assumptions $\frac 1{.13}<n$. Therefore there exists a natural number $M$ that $n>M$ as long as there exists a natural number $m\le \frac 1{.13}$ which is true since $7\le 1{.13}$. Thus we've proved there exists natural numbers $M$ and $n$ that satisfy $n>M$ and $\frac 1n < .13$.
    \end{proof}

    \item[e.] \begin{proof}
        Assume there exists a natural number $n$. If this is the case, there will also exist a natural number $k=n+1$ which implies that $k>n$. Thus we've proved that there exists no largest natural number since there will always exist a natural number $k$ that is $1$ greater the natural number $n$.
    \end{proof}

    \item[f.] \begin{proof}
        Assume there exists a positive real number $x$ such that $x>0$. If this is the case, there will also exist a positive real number $y=\frac x2$ which implies $y<x$ since both $x$ and $y$ are positive. Thus we've proved that there exists no largest natural number since there will always exist a real number $y$ that is half the size of any real number $x$.
    \end{proof}

    \item[i.] \begin{proof}
        We'll prove there exists a natural number $K$ that for all greater real numbers $r$, so that $r>K$, then $\frac 1{r^2}<0.01$. $\frac 1{r^2}<0.01\implies 100<r^2\implies 10<r\lor -10>r$. This proves that $K=10$, therefore if $r>K=10$ then $\frac 1{r^2}<0.01$. Thus we've proved that there exists a natural number $K$ that for all greater real numbers $r$ then $\frac 1{r^2}<0.01$.
    \end{proof}

    \item[k.] \begin{proof}
        We'll prove there exists an odd integer $M$, so that $M=2k+1$ where $k\in\mathbb{Z}$, that for all greater real numbers $r$, so that $r>M$, then $\frac1{2r}<0.01$. Since $\frac1{2r}<0.01$ then $100<2r\implies 50<r$ and $50\le51$ then let $M=51$ as $51=2(25)+1$. Then all real numbers $r$ greater than $M=51$ is implicitly greater than $50$ fulfilling $50<r$ therefore implying $\frac1{2r}<0.01$. Thus we've proved there exists an odd integer $M$ that for all greater real numbers $r$ then $\frac1{2r}<0.01$.
    \end{proof}

\end{itemize}

\item[2.1.4]
\begin{itemize}
    \item[a.] False

    \item[b.] True 

    \item[c.] False

    \item[d.] True

    \item[e.] True

    \item[f.] False

    \item[g.] True

    \item[h.] False

    \item[i.] False

    \item[j.] True

\end{itemize}

\item[2.1.5]
\begin{itemize}
    \item[a.] True

    \item[b.] True

    \item[c.] True

    \item[d.] True

    \item[e.] False

    \item[f.] True

    \item[g.] False

    \item[h.] False

    \item[i.] False

    \item[j.] False

    \item[k.] True

    \item[l.] True
    
\end{itemize}

\item[2.1.6]
\begin{itemize}
    \item[a.] $A\subseteq B, B\nsubseteq C,\text{ and }A\subseteq C$ \\
              $A=\{1\},B=\{1,2\},C=\{1,3\}$

    \item[b.] $A\subseteq B,B\subseteq C,\text{ and }C\subseteq A$ \\
              $A=\{1\},B=\{1\},C=\{1\}$

    \item[c.] $A\nsubseteq B, B\nsubseteq C,\text{ and }A\subseteq C$ \\
              $A=\{1\},B=\{2\},C=\{1,3\}$

    \item[d.] $A\subseteq B, B\nsubseteq C,\text{ and }A\nsubseteq C$ \\
              $A=\{1\},B=\{1,3\},C=\{2\}$

\end{itemize}

\item[2.1.8] \begin{proof}
    Assume the sets exist $A$, $B$, and $C$ and let $A\subseteq B$ and $B\subseteq C$. Therefore all elements $x\in A$ and because $A\subseteq B$ all elements $x\in B$. Finally, since $B\subseteq C$ makes all all elements $x\in C$. Since $A$ is entirely made up of all elements $x$ and $x\in C$ then $A\subseteq C$. Thus we've shown $A\subseteq C$ if $A\subseteq B$ and $B\subseteq C$.
\end{proof}

\item[2.1.11]
\begin{itemize}
    \item[a.] \begin{proof}
        Assume for all real numbers $x$ that $\frac{3x}{4}-2>10\iff \frac{3x}{4}>12\iff x>\frac{48}{3}\iff x>16\iff x\in(16,\infty ).$
    \end{proof}

    \item[b.] \begin{proof}
        We'll prove that $\{x\in\mathbb{R},|x-4|=2|x|-2\}=\{-6,2\}$ by showing $\{x\in\mathbb{R},|x-4|=2|x|-2\}\subseteq\{-6,2\}$ and $\{x\in\mathbb{R},|x-4|=2|x|-2\}\supseteq\{-6,2\}$.
        Assume that all real numbers $x$ satisfy $|x-4|=2|x|-2\implies (x-4)^2=4(|x|-1)^2
        \implies x^2-8x+16=4x^2-8|x|+4\implies 0=3x^2-8|x|+8x-12$.
        \paragraph{Case 1:}
            if $x\ge 0$ then $0=3x^2-12$ and therefore $0=x^2-4\implies x=2$. $x$ can only be $2$ for this case
        \paragraph{Case 2:}
            if $x< 0$ then $0=3x^2+16x-12$ and therefore $x=\frac{-16\pm\sqrt{16^2-4(3)(-12)}}6
            =\frac{-16\pm\sqrt{400}}{6}=-\frac{8}{3}\pm \frac{10}{3}=-6\text{ or }\frac 23$. Thus $x$ can only be equal to $6$ for this case.
        Thus we've shown that the only real numbers $x$ that satisfy satisfy $|x-4|=2|x|-2$ are all of the elements of $\{-6,2\}.$ Therefore we've proved that $\{x\in\mathbb{R},|x-4|=2|x|-2\}=\{-6,2\}$.
    \end{proof}

\end{itemize}

\item[2.1.14]
\begin{itemize}
    \item[a.] $\mathcal{P}(\{0,\triangle,\square\})=
    \{\varnothing,\{0\},\{\triangle\},\{\square\},\{0,\triangle\},\{0,\square\},\{\triangle,\square\},\{0,\triangle,\square\}\}$

    \item[b.] $\mathcal{P}(\{S,\{S\}\})=\{\varnothing,\{S\},\{\{S\}\},\{S,\{S\}\}\}$

\end{itemize}

\item[2.1.15]
\begin{itemize}
    \item[a.] False

    \item[b.] True

    \item[c.] False

    \item[d.] True

\end{itemize}

\end{itemize}

\end{document}