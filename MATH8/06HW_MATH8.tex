\documentclass[11pt]{amsart}

\usepackage{amssymb}

\usepackage{amsmath}

\usepackage{amsthm}

\usepackage{mathrsfs}

\usepackage[pdftex]{graphicx}

\tolerance = 10000

\setlength{\oddsidemargin}{8mm}

\setlength{\evensidemargin}{8mm}
 
\setlength{\topmargin}{-5mm}

\setlength{\textwidth}{145mm}

\setlength{\textheight}{220mm}

\setlength{\parindent}{0pt}

\setlength{\parskip}{1.5ex plus 0.5ex minus 0.2ex}

\newtheorem{theorem}{Theorem}[section]

\newtheorem{lemma}[theorem]{Lemma}

\newtheorem{corollary}[theorem]{Corollary}

\newtheorem{proposition}[theorem]{Proposition}

\newtheorem{conjecture}[theorem]{Conjecture}

\theoremstyle{definition}

\newtheorem*{example}{Example}

\newtheorem*{definition}{Definition}

%\renewcommand\qedsymbol{$\blacksquare$}

\newcommand*{\lxor}{\veebar}



\begin{document}

\title{Homework 6}

\author{Jesse Cobb - 2PM Section}

\maketitle

\begin{itemize}

\item[5.1.1] \begin{proof}
	Let $A$ be a set then $A\approx A$ since there exists a bijection $I_A:A\to A$ given by $I_A(x)=x$. Thus $\approx$ is a reflexive relation. \\
	Let $A$ and $B$ be sets and $A\approx B$ so there exists a bijection $f:A\to B$. Then by definition there exists an inverse bijection $f^{-1}:B\to A$ so $B\approx A$. Thus $\approx$ is a symmetric relation. \\
	Now Let $A$,$B$, and $C$ be sets and let $A\approx B$ and $B\approx C$ so that there exists bijections $f:A\to B$ and $g:B\to C$. By definition there exists a bijection $g\circ f:A\to C$ that is a composite of two bijections. This implies $A\approx C$ so $\approx$ is a transitive relation. \\
	Thus we've shown $\approx$ to be an equivalence relation as it is reflexive, symmetric, and transitive.
\end{proof}

\item[5.1.2]
\begin{itemize}
	\item[a.] $A=\{1,2,4,8,16,32,64,128,256,512\}\approx\mathbb{N}_{10}$\\
		$f:\mathbb{N}_{10}\to A$ where $f(n)=2^{n-1}$ so $\overline{\overline A}=10$

	\item[c.] $B=\{x\in\mathbb{Z}:x^2<11\}\approx\mathbb{N}_7$ \\
		$g:\mathbb{N}_7\to B$ where $g(n)=\begin{cases}
			\frac n2 & 2\mid n\\
			-\frac{n-1}2 & 2\nmid n
		\end{cases}$ so $\overline{\overline B}=7$

	\item[d.] $C=\{(x,y)\in\mathbb{N}:x+y<6\}\approx\mathbb{N}_{10}$\\
		$h:\mathbb{N}_{10}\to C$ where $h(1,2,3,4,5,6,7,8,9,10)\\=(1,1),(1,2),(1,3),(1,4),(2,1),(2,2),(2,3),(3,1),(3,2),(4,1)$ so $\overline{\overline C}=10$
\end{itemize}

\item[5.1.13] \begin{proof}
	
\end{proof}

\item[5.1.14]

\item[5.2.2]
\begin{itemize}
    \item[a.]

    \item[b.]

\end{itemize}

\item[5.2.3]
\begin{itemize}
    \item[c.]

    \item[e.]

    \item[f.]
    
\end{itemize}

\item[5.2.4]
\begin{itemize}
    \item[a.]

    \item[b.]

\end{itemize}

\item[5.3.1]

\item[5.3.4]

\item[5.3.13]
\begin{itemize}
    \item[a.]

\end{itemize}

\item[5.4.1]
\begin{itemize}
    \item[a.]

    \item[b.]

    \item[c.]

    \item[d.]
    
\end{itemize}

\item[5.4.5]

\end{itemize}

\end{document}
