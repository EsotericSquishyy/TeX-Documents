\documentclass[11pt]{amsart}

\usepackage{amssymb}

\usepackage{amsmath}

\usepackage{amsthm}

\usepackage{mathrsfs}

\usepackage[pdftex]{graphicx}

\tolerance = 10000

\setlength{\oddsidemargin}{8mm}

\setlength{\evensidemargin}{8mm}
 
\setlength{\topmargin}{-5mm}

\setlength{\textwidth}{145mm}

\setlength{\textheight}{220mm}

\setlength{\parindent}{0pt}

\setlength{\parskip}{1.5ex plus 0.5ex minus 0.2ex}

\newtheorem{theorem}{Theorem}[section]

\newtheorem{lemma}[theorem]{Lemma}

\newtheorem{corollary}[theorem]{Corollary}

\newtheorem{proposition}[theorem]{Proposition}

\newtheorem{conjecture}[theorem]{Conjecture}

\theoremstyle{definition}

\newtheorem*{example}{Example}

\newtheorem*{definition}{Definition}

%\renewcommand\qedsymbol{$\blacksquare$}

\newcommand*{\lxor}{\veebar}



\begin{document}

\title{Homework 3}

\author{Jesse Cobb - 2PM Section}

\maketitle

\begin{itemize}

\item[2.2.1] $A=\{1,3,5,7,9\}, B=\{0,2,4,6,8\},\\ C=\{1,2,4,5,7,8\}, D=\{1,2,3,5,6,7,8,9,10\}$
\begin{itemize}
    \item[d.] $A-(B-C)=A-\{0,6\}=\{1,3,5,7,9\}=A$

    \item[f.] $A\cup(C\cap D)=A\cup \{1,2,5,7,8\}=\{1,2,3,5,7,8,9\}$

    \item[h.] $A\cap (B\cup C)=A\cap \{0,1,2,4,5,6,7,8\}=\{1,5,7\}$
    
\end{itemize}

\item[2.2.2] $A=[3,8),B=[2,6],C=(1,4),D=(5,\infty)$
\begin{itemize}
    \item[b.] $A\cup B=[2,8)$

    \item[d.] $A\cap B=[3,6]$

    \item[f.] $A-B=(6,8)$
    
\end{itemize}

\item[2.2.5] $A=[3,8),B=[2,6],C=(1,4),D=(5,\infty)$\\
             $C\cap D=\varnothing$\\
             $C$ and $D$ is the only disjoint pair

\item[2.2.6]
\begin{itemize}
    \item[d.] $A\nsubseteq B\cup C, B\nsubseteq A\cup C,\text{ and }C\subseteq A\cup B$ \\
              $A=\{1,4\}, B=\{2,3\},C=\{1,2\}$

\end{itemize}

\item[2.2.7]
\begin{itemize}
    \item[q.] Prove if $A\subseteq B$, then $A\cup C\subseteq B\cup C$.
    \begin{proof}
        Assume all elements $x\in A$, and since $A\subseteq B$ then $x\in B$. By the definition of union $x\in A\cup C$ for any set $C$ with the addition of all elements $y\in C-A$. Since $x\in B$ then all $x\in B\cup C$ and all $y\in C$ so all $y\in B\cup C$. Since the elements $x$ and $y$ make up all of $A\cup C$ and $x,y\in B\cup C$ we've proved that  $A\cup C\subseteq B\cup C$ if $A\subseteq B$.
    \end{proof}

\end{itemize}

\item[2.2.8] 
\begin{itemize}
    \item[h.] Prove $(A\cap B)^c=A^c\cup B^c$.
    \begin{proof}
        \begin{align*}
            &x\in(A\cap B)^c &\iff \\
            &x\notin A\cap B &\iff \\
            &x\notin A\text{ or }x\notin B &\iff \\
            &x\in A^c\text{ or }x\in B^c &\iff \\
            &x\in A^c\cup B^c
        \end{align*}
        Since the last statement is true, so are all previous statements. Thus $(A\cap B)^c$ if and only if $A^c\cup B^c$.
    \end{proof}

\end{itemize}

\item[2.2.9]
\begin{itemize}
    \item[b.] \begin{proof}
        Assume for all $x\in A$ and $A\subseteq B\cup C$ and $A\cap B=\varnothing$, which implies $x\in B\cup C$ and $x\notin B$. Since $x\notin B$ and $x$ is in the union of $B\cup C$ this implies $x\in C$. Thus we've proved that $A\subseteq C$ if $A\subseteq B\cup C$ and $A\cap B=\varnothing$.
    \end{proof}

\end{itemize}

\item[2.2.10]
\begin{itemize}
    \item[a.] \begin{proof}
        Assume $x\in C$, $y\in D$, $C\subseteq A$, and $D\subseteq B$. This implies $x\in A$ and $y\in B$. By definition of intersection $z\in C\cap D$ where $z=x=y$. $z\in A\cap B$ since $x\in A$ and $y\in B$. Thus we've proved that $C\cap D\subseteq A\cap B$.
    \end{proof}

    \item[d.] \begin{proof}
        Assume $x\in C$, $y\in D$, $C\subseteq A$, and $D\subseteq B$. This implies $x\in A$ and $y\in B$. $D-A$ includes all elements $y$ that are not elements of $A$ which includes elements $x$. $B-C$ includes all elements $y$ that are not in $C$ which is only elements $x$. In summary $B-C$ includes $y$, not $x$, and any elements $z\in B-D$. While $D-A$ includes only $y$, not $x$, and not $w\in A-C$. Thus we've proved $D-A\subseteq B-C$ if $C\subseteq A$ and $D\subseteq B$.
    \end{proof}

\end{itemize}

\item[2.2.11]
\begin{itemize}
    \item[b.] Statement: if $A\cap C\subseteq B\cap C$ then $A\subseteq C$ \\
              Counterexample: $A=\{4\},B=\varnothing,C=\varnothing$

    \item[f.] Statement: $A-(B-C)=(A-B)-C$ \\
              Counterexample: $A=\{4,13\},B=\varnothing,C=\{4\}$

\end{itemize}

\item[2.2.13]
\begin{itemize}
    \item[a.] $A=\{1,3,4\}, B=\{a,e,k,n,r\}$ \\
              $A\times B=\{(1,a),(1,e),(1,k),(1,n),(1,r) \\
                           (3,a),(3,e),(3,k),(3,n),(3,r) \\
                           (5,a),(5,e),(5,k),(5,n),(5,r)\}$ \\
              $B\times A=\{(a,1),(a,3),(a,5),(e,1),(e,3),(e,5) \\
                           (k,1),(k,3),(k,5),(n,1),(n,3),(n,5) \\
                           (r,1),(r,3),(r,5)\}$
    
\end{itemize}

\item[2.2.15]
\begin{itemize}
    \item[b.] \begin{proof}
        Seeking a contradiction, assume $A\times \varnothing\ne \varnothing$. Assuming this there must be at least one element $x\in A\times \varnothing$. Which implies the existence of the ordered pair $(a,b)$ where $a\in A$ and $b\in\varnothing$, though this is a contradiction as there can be no $b\in\varnothing$. Thus we've proved by contradiction that $A\times \varnothing=\varnothing$.
    \end{proof}
    
\end{itemize}

\item[2.2.16]
\begin{itemize}
    \item[a.] $(A\times B)\cup(C\times D)\ne(A\cup B)\times(C\cup D)$ \\
              $A=\{1\},B=\{2\},C=\{3\},D=\{4\}$
    
\end{itemize}

\item[2.3.1]
\begin{itemize}
    \item[b.] $\mathscr{A}=\{\{1,3,5\},\{2,4,6\},\{7,9,11,13\},\{8,10,12\}\}$ \\
              $\displaystyle
              \bigcup_{A\in\mathscr{A}}A=\{n\in\mathbb{N}:n\le 13\}
              \qquad
              \bigcap_{A\in\mathscr{A}}A=\varnothing$

    \item[h.] $\mathscr{A}=\{A_r=[-\pi,r):r\in(0,\infty)\}$ \\
              $\displaystyle
              \bigcup_{A\in\mathscr{A}}A=[-\pi,\infty)
              \qquad
              \bigcap_{A\in\mathscr{A}}A=[-\pi,0]$

    \item[l.] $\mathscr{C}=\{C_n=[n,n+1):n\in\mathbb{Z}\}$ \\
              $\displaystyle
              \bigcup_{C\in\mathscr{C}}C=(-\infty,\infty)
              \qquad
              \bigcap_{C\in\mathscr{C}}C=\varnothing$

    \item[m.] $\mathscr{A}=\{A_n=(n,n+1):n\in\mathbb{Z}\}$ \\
              $\displaystyle
              \bigcup_{A\in\mathscr{A}}A=\{x\in\mathbb{R}:x\notin\mathbb{Z}\}
              \qquad
              \bigcap_{A\in\mathscr{A}}A=\varnothing$

    \item[n.] $\mathscr{D}=\{D_n=(-n,\frac 1n):n\in\mathbb{N}\}$ \\
              $\displaystyle
              \bigcup_{D\in\mathscr{D}}D=(-\infty,1)
              \qquad
              \bigcap_{D\in\mathscr{D}}D=(-1,0]$
    
\end{itemize}

\item[2.3.2]
\begin{itemize}
    \item[b.] Pairwise Disjoint

    \item[h.] Not Pairwise Disjoint

    \item[l.] Pairwise Disjoint

    \item[m.] Pairwise Disjoint

    \item[n.] Not Pairwise Disjoint
    
\end{itemize}

\item[2.3.3]
\begin{itemize}
    \item[a.] For every set $B$ in the family $\mathscr{A}$, $B\subseteq \bigcup_{A\in\mathscr{A}}A$.
    \begin{proof}
        Let $B\in\mathscr{A}$ and $x\in B$. Since $B\in\mathscr{A}$, by the definition of union $x\in\bigcup_{A\in\mathscr{A}}A$. Thus we've proved $B\subseteq\bigcup_{A\in\mathscr{A}}A$ if $B\in\mathscr{A}$.
    \end{proof}

    \item[b.] If $A\subseteq B$ for all $A\in\mathscr{A}$, then $\bigcup_{A\in\mathscr{A}}A\subseteq B$.
    \begin{proof}
        Let $A\subseteq B$ for any set $A\in\mathscr{A}$. If $x\in\bigcup_{A\in\mathscr{A}}A$ then $x\in B$. Thus we've proved that $\bigcup_{A\in\mathscr{A}}A\subseteq B$ if $A\subseteq B$ for all $A\in\mathscr{A}$.
    \end{proof}

\end{itemize}

\item[2.3.12] $X=\{1,2,3,4,\ldots20\}$
\begin{itemize}
    \item[a.] $\displaystyle
              \bigcup_{A\in\mathscr{A}}A=X
              \qquad
              \bigcap_{A\in\mathscr{A}}A=\{1\} \\
              \mathscr{A}=\{\{1\},X\}$
    
    \item[b.] $\displaystyle
              \bigcup_{B\in\mathscr{B}}B=X\qquad\text{4 disjoint subsets of }X\\
              \mathscr{B}=\{\{1,2,3,4,5\},\{6,7,8,9,10\},\{11,12,13,14,15\},\{16,17,18,19,20\}\}$
    
    \item[c.] $\displaystyle
              \bigcup_{C\in\mathscr{C}}C=X\qquad\text{20 disjoint subsets of }X\\
              \mathscr{C}=\{\{1\},\{2\},\{3\},\{4\},\{5\},\{6\},\{7\},\{8\},\{9\},\{10\},\\\{11\},\{12\},\{13\},\{14\},\{15\},\{16\},\{17\},\{18\},\{19\},\{20\}\}$

\end{itemize}

\item[2.3.16] $\mathscr{A}=\{A_i:i\in\mathbb{N}\}\quad k,m\in\mathbb{N}\quad k\le m$
\begin{itemize}
    \item[d.] $\displaystyle\bigcap_{i=1}^{\infty}A_i\subseteq \bigcap_{i=k}^{m}A_i\ $
    \begin{proof}
        Let $x\in\bigcap_{i=1}^{\infty}A_i$ which implies $x\in A_i$ for any $A_i\in\mathscr{A}$. This implies that $x\in\bigcap_{i=k}^{m}A_i$ since $k\ge 1$ and $m<\infty$. Thus we've proved that $\bigcap_{i=1}^{\infty}A_i\subseteq \bigcap_{i=k}^{m}A_i$ if $\mathscr{A}=\{A_i:i\in\mathbb{N}\}$, and $k,m\in\mathbb{N}$, and $k\le m$.
    \end{proof}
    
\end{itemize}

\item[2.3.18] Nested family $\mathscr{A}=\{A_i:i\in\mathbb{N}\}$
\begin{itemize}
    \item[b.] $\bigcap_{i=1}^\infty A_i=(-\infty,1]\implies A_i=(-\infty,1+\frac1i)$
    
    \item[d.] $\bigcap_{i=1}^\infty A_i=\varnothing\implies A_i=(0,\frac1n)$
    
\end{itemize}

\item[2.4.4]
\begin{itemize}
    \item[b.]\begin{proof}
        We’ll prove the claim using mathematical induction. First note that when $n=1$ that $3=4(1)^2-1$ is true, which proves the base case holds. Next assume $3+11+19+\ldots(8n-5)=4n^2-n$ for all $n\in\mathbb{N}$. Then $3+11+19+\ldots(8n-5)+(8(n+1)-5)=4n^2-n+(8(n+1)-5)$ by the inductive hypothesis. This implies $4n^2-n+(8(n+1)-5)=4n^2+7n+3=4(n+1)^2-(n+1)$ so the statement works for the $n+1$ case. Thus we've proved by the Principle of Mathematical Induction, that $3+11+19+\ldots(8n-5)=4n^2-n$ for all natural numbers $n$.
    \end{proof}
    
    \item[c.]\begin{proof}
        We’ll prove the claim using mathematical induction. First note that when $n=1$ that $\sum_{i=1}^n2^i=2^{n+1}-2\implies 2^1=2^{2}-2$ which proves the base case holds. Now assume $\sum_{i=1}^n2^i=2^{n+1}-2$ for all natural numbers $n$. Then $\sum_{i=1}^n2^i+2^{n+1}=2^{n+1}-2+2^{n+1}$ by the inductive hypothesis. Thus $2^{n+1}-2+2^{n+1}=2\cdot 2^{n+1}-2=2^{(n+1)+1}-2$ which proves that the $n+1$ case is true. Thus we've proved, by the Principle of Mathematical Induction, that $\sum_{i=1}^n2^i=2^{n+1}-2$ for all $n\in\mathbb{N}$.
    \end{proof}

    \item[d.]\begin{proof}
        We’ll prove the claim using mathematical induction. First note that when $n=1$ that $1\cdot1!+2\cdot2!+3\cdot3!+\ldots n\cdots n!=(n+1)!-1\implies 1\cdot 1!=2!-1$ so the base case holds. Now assume $1\cdot1!+2\cdot2!+3\cdot3!+\ldots n\cdot n!=(n+1)!-1$ for all $n\in\mathbb{N}$. Then $1\cdot1!+2\cdot2!+3\cdot3!+\ldots n\cdot n!+(n+1)\cdot(n+1)!=(n+1)!-1+(n+1)\cdot(n+1)!$ by the induction hypothesis. Then $(n+1)!-1+(n+1)\cdot(n+1)!=(n+2)(n+1)!-1=(n+2)!-1$ which proves the $n+1$ case. Thus we've proved, by the Principle of Mathematical Induction, that $1\cdot1!+2\cdot2!+3\cdot3!+\ldots n\cdots n!=(n+1)!-1$ for all natural numbers $n$.
    \end{proof}
    
    \item[e.]\begin{proof}
        We’ll prove the claim using mathematical induction. First note that when $n=1$ that $1^3+2^3+3^3+\ldots n^3=[\frac{n(n+1)}2]^2\implies 1^3=[\frac22]^2$ which proves the base case is true. Now assume $1^3+2^3+3^3+\ldots n^3=[\frac{n(n+1)}2]^2$ for all $n\in\mathbb{N}$. Then $1^3+2^3+3^3+\ldots n^3+(n+1)^3=[\frac{n(n+1)}2]^2+(n+1)^3$ by the induction hypothesis. Then: \begin{align*}
            [\frac{n(n+1)}2]^2+(n+1)^3&=
            \frac{(n(n+1))^2+4(n+1)^3}{4} \\
            &=\frac{n^2(n+1)^2+4(n+1)(n+1)^2}{4} \\
            &=\frac{(n+1)^2(n^2+4n+4)}{4}\\
            &=\frac{(n+1)^2(n+2)^2}{4}\\
            &=[\frac{(n+1)(n+2)}2]^2
        \end{align*}
        which proves the $n+1$ case. Thus we've proved, by the Principle of Mathematical Induction, that $1^3+2^3+3^3+\ldots n^3=[\frac{n(n+1)}2]^2$ for all natural numbers $n$.
    \end{proof}
    
\end{itemize}

\item[2.4.5]
\begin{itemize}
    \item[a.]\begin{proof}
        We’ll prove the claim using mathematical induction. First note that when $n=1$ that $3\mid n^3+5n+6\implies 3\mid 1^3+5(1)+6$ which is true since $1^3+5(1)+6=12=3(4)$ where $4\in\mathbb{Z}$ which proves the base case. Now assume $3\mid n^3+5n+6$ which implies $n^3+5n+6=3k$ for $n\in\mathbb{N}$ and $k\in\mathbb{Z}$. Then $(n+1)^3+5(n+1)+6=n^3+3n^2+8n+12=(n^3+5n+6)+3n^2+3n+6=3k+3n^2+3n+6$ by the induction hypothesis. Then $3k+3n^2+3n+6=3(k+n^2+n+2)=3j$ where $j=k+n^2+n+2$ is an integer. This proves that the $n+1$ case is true. Thus we've proved, by the Principle of Mathematical Induction, that $3\mid n^3+5n+6$ for all natural numbers $n$.
    \end{proof}
    
    \item[j.]\begin{proof}
        We’ll prove the claim using mathematical induction. First note that when $n=1$ that $3^n\ge 1+2^n\implies 3^1\ge 1+2^n$ which proves the base case. Now assume $3^n\ge 1+2^n$ for all natural numbers $n$. Then $3^{n+1}=3\cdot 3^n\ge 3(1+2^n)$ by the induction hypothesis. Then $3(1+2^n)=(1+2^n)+(1+2^n)+(1+2^n)=3+2^n+2\cdot2^n=3+2^n+2^{n+1}>1+2^{n+1}$ which implies that $3^{n+1}\ge 1+2^{n+1}$ which proves the $n+1$ case. Thus we've proved, by the Principle of Mathematical Induction, that $3^n\ge 1+2^n$ for all natural numbers $n$.
    \end{proof}

    \item[q.]\begin{proof}
        We’ll prove the claim using mathematical induction. First note that when $n=1$ that $|A|=1$ so that $A=\{a\}$, then the power set $\mathscr{P}(A)=\{\varnothing,\{a\}\}$ which implies $|\mathscr{P}(A)|=2$ which proves the base case true as $|\mathscr{P}(A)|=2^{|A|}=2$. Now assume that if $|A|=n$ then $|\mathscr{P}(A)|=2^n$ for all natural numbers $n$. Now let the set $B$ have $n+1$ elements where there exists an element $x\in B$ but $x\notin A$. Then $\mathscr{P}(B)$ has $2^n$ elements excluding $x$ and $2^n$ elements including $x$ by the inductive hypothesis, so $|\mathscr{P}(B)|=2\cdot2^n=2^{n+1}$, so we've proved the $n+1$ case. Thus we've proved, by the Principle of Mathematical Induction, that $\mathscr{P}(A)$ has $2^n$ elements if $A$ has $n$ elements for all $n\in\mathbb{N}$.
    \end{proof}
    
\end{itemize}

\item[2.4.6]
\begin{itemize}
    \item[c.]\begin{proof}
        We’ll prove the claim using mathematical induction. First note that when $n=5$ that $(n+1)!>2^{n+3}\implies 6!>2^8$ which is true since $720>256$ thus proving the base case. Now assume $(n+1)!>2^{n+3}$ for all natural numbers $n\ge 5$. Then:\begin{align*}
            (n+2)!
            &=(n+2)(n+1)! \\
            &>(n+2)2^{n+3} \\
            &>2\cdot2^{n+3} \\
            &=2^{n+4}
        \end{align*}
        which proves $(n+2)!>2^{n+4}$ by the induction hypothesis, proving the $n+1$ case true. Thus we've proved, by the Principle of Mathematical Induction, that $(n+1)!>2^{n+3}$ for all natural numbers $n\ge 5$.
    \end{proof}
    
    \item[e.]\begin{proof}
        We’ll prove the claim using mathematical induction. First note that when $n=4$ that $n!>3n\implies 4!>3(4)$ which is true since $24>12$ which proves the base case. Now assume $n!>3n$ for all natural numbers $n\ge 4$. Then:\begin{align*}
            (n+1)!
            &=(n+1)n! \\
            &>(n+1)3n \\
            &>3(n+1)
        \end{align*}
        which proves $(n+1)!>3(n+1)$ by the induction hypothesis, proving the $n+1$ case true. Thus we've proved, by the Principle of Mathematical Induction, that $n!>3n$ for all natural numbers $n\ge 4$.
    \end{proof}
    
\end{itemize}

\item[2.4.7]
\begin{itemize}
    \item[a.]\begin{proof}
        We’ll prove the claim using mathematical induction. First note that when $n=1$ that $(\bigcap_{i=1}^nA_i)^c=\bigcup_{i=1}^nA_i^c\implies (A_i)^c=A_i^c$ which proves the base case. Now assume $(\bigcap_{i=1}^nA_i)^c=\bigcup_{i=1}^nA_i^c$ for all $n\in\mathbb{N}$ where the indexed family $\{A_i:i\in\mathbb{N}\}$ exists. Then:\begin{align*}
            \left(\bigcap_{i=1}^{n+1}A_i\right)^c
            &=\left(\bigcap_{i=1}^nA_i\cap A_{n+1}\right)^c\\
            &=\left(\bigcap_{i=1}^nA_i\right)^c\cup A_{n+1}^c\\
            &=\bigcup_{i=1}^nA_i^c\cup A_{n+1}^c\\
            &=\bigcup_{i=1}^{n+1}A_i^c
        \end{align*}
        which proves $(\bigcap_{i=1}^{n+1}A_i)^c=\bigcup_{i=1}^{n+1}A_i^c$ by the induction hypothesis, proving the $n+1$ case true. Thus we've proved, by the Principle of Mathematical Induction, that $(\bigcap_{i=1}^nA_i)^c=\bigcup_{i=1}^nA_i^c$ for all natural numbers $n$.
    \end{proof}
    
\end{itemize}

\item[2.4.9]\begin{proof}
    We’ll prove the claim using mathematical induction. First note that when $n=1$ that $1$ point requires $\frac{1^2-1}2=0$ points, which proves the base case. Now assume that for $n$ points, with no three collinear points, the amount of line segments to join all of the points is $\frac{n^2-n}2$ for all natural numbers $n$. Then $n+1$ points requires an additional $n$ points since the new point will have to connect to each $n$ previously existing points. Then $n+1$ points requires $\frac{n^2-n}2+n=\frac{n^2-n+2n}2=\frac{(n+1)^2-(n+1)}2$ points which proves the $n+1$ case by the induction hypothesis. Thus we've proved, by the Principle of Mathematical Induction, that for any natural number $n$ points, where no three points are collinear, they require $\frac{n^2-n}2$ line segments to connect all points.
\end{proof}

\item[2.4.10]\begin{proof}
    We’ll prove the claim using mathematical induction. First note that when $n=1$ that $2^{(1)}-1=1$ implies that with $1$ disc it takes a minimum of $1$ move to the disc to another peg without any larger discs ever being on top of a smaller disc. This is true as any possible move will achieve the desired result with a single disc confirming our base case. Now assume that any natural number $n$ of discs can achieve the same result of being stacked in descending size on another peg then its starting position in $f(n)=2^n-1$ moves, this implies the recursive formula of $f(n)=2f(n-1)+1$ as each additional disc requires $1$ additional move to move the new disc and another $f(n-1)$ moves to then move the stack back on top of the new largest disc. Then $f(n+1)=2f(n)+1=2(2^n-1)+1$ by the induction hypothesis. Then $2(2^n-1)+1=2\cdot2^n-2+1=2^{n+1}-1$ which proves the $n+1$ case. Thus we've proved, by the Principle of Mathematical Induction, that for any natural number $n$ of disc the discs can all be moved to a new peg (out of three), without ever having a larger disc above a smaller disc, in $2^n-1$ moves.
\end{proof}

\item[2.5.1]
\begin{itemize}
    \item[a.]\begin{proof}
        We’ll prove the claim using complete induction. Suppose $n\ge 11$, then note that $11=2(3)+5(1)$ and $12=2(1)+5(2)$ verifying our base cases. Now we'll assume that $n\ge 13$, and that all natural numbers $k$ where $n-1\ge k\ge 11$ can be written as $k=2s+5t$ for some $s,t\in\mathbb{N}$. We'll demonstrate that the same is true for $n$. Since $n\ge 13$ then $n-2\ge 11$, so $n-2=2s+5t$ by assumption. Then $n=2s+5t+2=2(s+1)+5t$ where $s+1,t\in\mathbb{N}$ and the statement is true for $n$. Thus we've proved, by the Principle of Complete Induction, that any natural number $n\ge 11$ can be written as $n=2s+5t$ for some $s,t\in\mathbb{N}$.
    \end{proof}
    
\end{itemize}

\item[2.5.10]\begin{proof}
    Let the set $\mathbb{Z}^-=\{-1,-2,-3\ldots\}$ exist and also allow the set $A\subseteq \mathbb{Z}^-$ to exist. Assume any set $A$ is nonempty, so that $A\ne \varnothing$, so that there exists an element $a\in A$. Now suppose a set $B$ exists where $B=\{-a:a\in A\}$. Since the elements of $A$ are all negative integers the elements of $B$ must be only positive integers meaning $B\subseteq \mathbb{N}$. Since $B\subseteq \mathbb{N}$ we may apply the Well Ordering Principle, since it applies to all subsets of $\mathbb{N}$, and say that there exists an element $b$ that is the smallest element $b\in B$ where $b\le -a$. Inversely this implies the existence of a $-b\in A$ where $-b\ge a$. Thus we've proved that any subset $A$ of $\mathbb{Z}^-$ must have a largest element as its inverse $B\subseteq\mathbb{N}$ must have a smallest element, by the Well Ordering Principle.
\end{proof}

\end{itemize}

\end{document}