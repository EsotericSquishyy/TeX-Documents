\documentclass[12pt, letterpaper]{article}
\usepackage{fullpage}
\usepackage[top=2cm, bottom=4.5cm, left=2.5cm, right=2.5cm]{geometry}
\usepackage{amsmath,amsthm,amsfonts,amssymb,amscd}
\usepackage{lastpage}
\usepackage{enumerate}
\usepackage{fancyhdr}
\usepackage{mathrsfs}
\usepackage{xcolor}
\usepackage{graphicx}
\usepackage{listings}
\usepackage{hyperref}
\usepackage{tikz}
\usetikzlibrary{decorations.markings}
\usepackage{mathtools}
\usepackage{breqn}
\usepackage{tensor}
\newenvironment{sol}
    {\emph{Solution:}
    }
    {
    %\qed
    }
\newcommand{\dbar}{d\hspace*{-0.08em}\bar{}\hspace*{0.1em}} %You can use \newcommand to create certain new commands, either abbreviations for other commands or even entirely new ones (like this one)
\newcommand{\Cl}{\text{Cl}}
\newcommand{\Int}{\text{Int}}
\newcommand{\sgn}{\text{sgn}}
\newcommand{\id}{\text{id}}
\newcommand{\im}{\text{Im}}
\newcommand{\dd}{\text{d}}
\newcommand{\tr}{\text{tr}}
\newcommand{\C}{\mathbb{C}}
\newcommand{\R}{\mathbb{R}}

%Various possibilities for "types" of question

\theoremstyle{plain}
\newtheorem{theorem}{Theorem}[section]
\newtheorem{thm}{Theorem}[section]
\newtheorem{coro}[theorem]{Corollary}
\newtheorem{cor}[theorem]{Corollary}
\newtheorem{prop}[theorem]{Proposition}
\newtheorem{lemma}[theorem]{Lemma}
\newtheorem{prob}[theorem]{Problem}

\theoremstyle{definition}
\newtheorem{definition}[theorem]{Definition}
\newtheorem{defn}[theorem]{Definition}
\newtheorem{example}[theorem]{Example}
\theoremstyle{remark}
\newtheorem*{remark}{Remark}
\newtheorem*{rmk}{Remark}
\newtheorem*{rem}{Remark}
\newtheorem*{eg}{e.g}
\newtheorem*{recall}{Recall}
\newtheorem*{fact}{Fact}
\newtheorem*{notation}{Notation}

\pagestyle{fancyplain}
\headheight 42pt
\chead{\textbf{\Large Homework 1}} %This one for assignment name
\rhead{09.30.23 \\ Jesse Cobb \\ Math 108A} %This one for date/name/class
\lfoot{}
\cfoot{}
\rfoot{\small\thepage}
\headsep 1.5em
\linespread{1.5}

\begin{document}

\section{Question 1} %Problem 1 goes within this section

\begin{prob} %Can use various commands depending on what exactly a given problem asks you to do
Using only the axioms of an abelian group, show that the identity element of an abelian group is unique. Show also that the inverse of any given element is unique.
\end{prob}

\begin{proof}
We will show that there exists a unique identity element, $e$ in any abelian group, $G$. Attempting to prove by contradiction, let $e'\in G$ be another identity element so that for all elements $x\in G$ satisfy $e'*x=x$ by the identity axiom. Since $e,e'\in G$ we can say $e'*e=e$ and $e*e'=e'$ by the identity axiom. By the commutative axiom $e*e'=e'\implies e'*e=e'$. Thus, since we've shown that $e'*e=e'$ and $e'*e=e$ we have proven the identity element $e$ to be unique as $e'=e$.
\end{proof}

\begin{proof}
We will show that the inverse element $y$ is unique given an element $x$ of an abelian group $G$. By the inverse axiom $y*x=e$ where $e\in G$ is the identity element of $G$. In an attempt to prove by contradiction assume there exists a second inverse element $y'\in G$ that satisfies $y'*x=e$ by the inverse axiom. Consider the following:
\begin{align*}
	y &= e*y 	&\text{(Identity)} \\
	  &= y'*x*y 	&\text{(Inverse)} \\
	  &= y'*(y*x)	&\text{(Commutative)} \\
	  &= y'*e	&\text{(Inverse)} \\
	  &= y'		&\text{(Identity)}
\end{align*}
Thus, we've shown that $y=y'$ which means that the the inverse element $y$ with respect to the element $x$ is unique.
\end{proof}


\section{Question 2} %Problem 2 goes within this section

\begin{prob}
Let $F$ be a field, $0_F\in F$ its additive identity. Let $V$ be a $F$-vector space and $0_V\in V$ denote its identity. Using only the axioms of a field and the axioms of a vector space, show that $$0_F\cdot v=0_V$$ for any $v\in V$. Here the dot denotes the scalar multiplication.
\end{prob}

\begin{proof}
We will show that any element $v\in V$ of $F$-vector space $V$ will be equal to $V$'s additive identity $0_V\in V$ if scalar multiplied by $F$'s additive identity $0_F\in F$ such that $0_F\cdot v=0_V$. Consider the following:
\begin{align*}
    0_F\cdot v &= (0_F+0_F)\cdot v      &\text{(Additive Identity of $F$)} \\
               &= 0_F\cdot v+0_F\cdot v &\text{(Distributive)} \\
\end{align*}
Now consider the following based on the previous proof that $0_F\cdot v=0_F\cdot v+0_F\cdot v$ and let $-(0_F\cdot v)\in V$ be the additive inverse of $0_F\cdot v$:
\begin{align*}
    0_F\cdot v=0_F\cdot v+0_F\cdot v &\implies -(0_F\cdot v)+0_F\cdot v=-(0_F\cdot v)+0_F\cdot v+0_F\cdot v \\
                                     &\implies 0_V=0_V+0_F\cdot v &\text{(Additive Inverse)} \\
                                     &\implies 0_V=0_F\cdot v     &\text{(Additive Identity)}
\end{align*}
Thus, we've shown that $0_F\cdot v=0_V$ for any element $v\in V$.
\end{proof}


\section{Question 3} %Problem n goes within this section

\begin{prob}
    Let $F$ be a field and $V$ be a vector space. Using only the axioms of a vector space, show that for any $\alpha\in F\backslash\{0\}$ and $u,v\in V$, there exists a unique $w\in V$ such that $$u+\alpha w=v.$$
\end{prob}

\begin{proof}
We will show that for any $\alpha\in F\backslash \{0\}$ and any $u,v\in V$ there exists a unique $w\in V$ that satisfies $u+\alpha w=v$, where $F$ is a field and $V$ is an $F$-vector space. Seeking a contradiction, let $w'$ be a different solution such that $u+\alpha w'=v$ and $w\ne w'$ so that $u+\alpha w=v=u+\alpha w'$. Assume $-u\in V$ is the additive inverse of $u$ and $1/\alpha \in F\backslash \{0\}$ is the multiplicative inverse of $\alpha$. Consider the following:
\begin{align*}
    u+\alpha w=u+\alpha w' &\implies (-u)+u+\alpha w=(-u)+u+\alpha w' \\
                           &\implies 0_V+\alpha w=0_V+\alpha w' &\text{(Additive Inverse)} \\
                           &\implies \alpha w=\alpha w' &\text{(Additive Identity)} \\
                           &\implies (1/\alpha)(\alpha w)=(1/\alpha)(\alpha w) \\
                           &\implies ((1/\alpha)\alpha) w=((1/\alpha)\alpha) w &\text{(Associativity)} \\
                           &\implies 1_Fw=1_Fw' &\text{(Multiplicative Inverse)} \\
                           &\implies w=w' &\text{(Multiplicative Identity)}
\end{align*}
Thus, we've shown that $w=w'$ which is a contradiction that proves that the solution $w\in V$ to $u+\alpha w=v$ is unique relative to the elements $u,v\in V$ and $\alpha \in F\backslash\{0\}$. 
\end{proof}


\section{Question 4} %Problem n goes within this section

\begin{prob}
	Let $\mathbb{C}$ be the field of complex numbers and $i\in\mathbb{C}$ be an element satisfying $i^2=-1$. Given $a,b\in\mathbb{R}$ not both zero, find $c,d\in\mathbb{R}$ such that $$\frac1{a+bi}=c+di.$$
\end{prob}

\begin{sol}
	\begin{align*}
		\frac1{a+bi} &= \frac1{a+bi}\left(\frac{a-bi}{a-bi}\right) \\
			     &= \frac{a-bi}{a^2+b^2} \\
			     &= \frac a{a^2+b^2}-\frac b{a^2+b^2}i \\
              &= c+di
	\end{align*}
 By equivalence of real and imaginary parts:
	$$c=\frac a{a^2+b^2}\qquad d=-\frac b{a^2+b^2}$$
\end{sol}


\section{Question 5} %Problem n goes within this section

\begin{prob}
Show that the set $\{0,1\}$ under addition and multiplication is a field where we require that $1+1=0$.
\end{prob}

\begin{proof}
We will show that for the set $X=\{0,1\}$ to be considered a field we require that $1+1=0$. The additive behavior of $0$ is determined by the axioms of a field so $0+1=1+0=1$ and $0+0=0$ by the additive identity axiom. By that same token the multiplicative behavior of $1$ is already determined $1\cdot 0=0\cdot 1=0$ and $1\cdot 1=1$ by the multiplicative identity axiom. We have shown $0\cdot a=0$ by Proof 2 as $X$ is an $X$-vector space. Finally we must find the behavior of $1+1\in X$. Seeking a contradiction let $1+1=1$. By adding the additive inverse to both sides we find $1+0=0\implies 1=0$, which results in a contradiction. Thus we've proven that the set $\{0,1\}$ is a field when $1+1=0$ by contradiction and by showing all other additive and multiplicative relationships operate as expected under the axioms of a field.
\end{proof}


\end{document}
