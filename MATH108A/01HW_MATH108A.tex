\documentclass[12pt, letterpaper]{article}
\usepackage{fullpage}
\usepackage[top=2cm, bottom=4.5cm, left=2.5cm, right=2.5cm]{geometry}
\usepackage{amsmath,amsthm,amsfonts,amssymb,amscd}
\usepackage{lastpage}
\usepackage{enumerate}
\usepackage{fancyhdr}
\usepackage{mathrsfs}
\usepackage{xcolor}
\usepackage{graphicx}
\usepackage{listings}
\usepackage{hyperref}
\usepackage{tikz}
\usetikzlibrary{decorations.markings}
\usepackage{mathtools}
\usepackage{breqn}
\usepackage{tensor}
\newenvironment{sol}
    {\emph{Solution:}
    }
    {
    %\qed
    }
\newcommand{\dbar}{d\hspace*{-0.08em}\bar{}\hspace*{0.1em}} %You can use \newcommand to create certain new commands, either abbreviations for other commands or even entirely new ones (like this one)
\newcommand{\Cl}{\text{Cl}}
\newcommand{\Int}{\text{Int}}
\newcommand{\sgn}{\text{sgn}}
\newcommand{\id}{\text{id}}
\newcommand{\im}{\text{Im}}
\newcommand{\dd}{\text{d}}
\newcommand{\tr}{\text{tr}}
\newcommand{\C}{\mathbb{C}}
\newcommand{\R}{\mathbb{R}}

%Various possibilities for "types" of question

\theoremstyle{plain}
\newtheorem{theorem}{Theorem}[section]
\newtheorem{thm}{Theorem}[section]
\newtheorem{coro}[theorem]{Corollary}
\newtheorem{cor}[theorem]{Corollary}
\newtheorem{prop}[theorem]{Proposition}
\newtheorem{lemma}[theorem]{Lemma}
\newtheorem{prob}[theorem]{Problem}

\theoremstyle{definition}
\newtheorem{definition}[theorem]{Definition}
\newtheorem{defn}[theorem]{Definition}
\newtheorem{example}[theorem]{Example}
\theoremstyle{remark}
\newtheorem*{remark}{Remark}
\newtheorem*{rmk}{Remark}
\newtheorem*{rem}{Remark}
\newtheorem*{eg}{e.g}
\newtheorem*{recall}{Recall}
\newtheorem*{fact}{Fact}
\newtheorem*{notation}{Notation}

\pagestyle{fancyplain}
\headheight 42pt
\chead{\textbf{\Large Homework 1}} %This one for assignment name
\rhead{09.20.23 \\ Jesse Cobb \\ Math 108A} %This one for date/name/class
\lfoot{}
\cfoot{}
\rfoot{\small\thepage}
\headsep 1.5em
\linespread{1.5}

\begin{document}

\section{Question 1} %Problem 1 goes within this section

\begin{prob} %Can use various commands depending on what exactly a given problem asks you to do
Using only the axioms of an abelian group, show that the identity element of an abelian group is unique. Show also that the inverse of any given element is unique. State clearly which axiom is being used whenever you use one at each step.
\end{prob}

\begin{proof}
We will show that there exists a unique identity element, $e$ in any abelian group, $G$. Attempting to prove by contradiction, let $e'\in G$ be another identity element so that for all elements $x\in G$ satisfy $e'*x=x$ by the identity axiom. Since $e,e'\in G$ we can say $e'*e=e$ and $e*e'=e'$ by the identity axiom. By the commutative axiom $e*e'=e'\implies e'*e=e'$. Thus, since we've shown that $e'*e=e'$ and $e'*e=e$ we have proven the identity element $e$ to be unique as $e'=e$.
\end{proof}

\begin{proof}
We will show that the inverse element $y$ is unique given an element $x$ of an abelian group $G$. By the inverse axiom $y*x=e$ where $e\in G$ is the identity element of $G$. In an attempt to prove by contradiction assume there exists a second inverse element $y'\in G$ that satisfies $y'*x=e$ by the inverse axiom. Consider the following:
\begin{align*}
	y &= e*y 	&\text{(Identity)} \\
	  &= y'*x*y 	&\text{(Inverse)} \\
	  &= y'*(y*x)	&\text{(Commutative)} \\
	  &= y'*e	&\text{(Inverse)} \\
	  &= y'		&\text{(Identity)}
\end{align*}
Thus, we've shown that $y=y'$ which means that the the inverse element $y$ with respect to the element $x$ is unique.

\end{proof}


\section{Question 2} %Problem 2 goes within this section

\begin{prob}
Let $F$ be a field, $0_F\in F$ its additive identity. Let $V$ be a $F$-vector space and $0_V\in V$ denote its identity. Using only the axioms of a field and the axioms of a vecotr space, show that $$0_F\cdot v=0_V$$ for any $v\in V$. Here the dot denotes the scalar multiplication. State clearly which axiom is being used whenever you use one at each step.
\end{prob}

\begin{proof}

\end{proof}

\section{Question 3} %Problem n goes within this section

\begin{prob}
Problem 3 statement here.
\end{prob}

\begin{sol}
Solution to problem 3 solution.
\end{sol}

\section{Question 4} %Problem n goes within this section

\begin{prob}
	Let $\mathbb{C}$ be the field of complex numbers and $i\in\mathbb{C}$ be an element satisfying $i^2=-1$. Given $a,b\in\mathbb{R}$ not both zero, find $c,d\in\mathbb{R}$ such that $$\frac1{a+bi}=c+di$$.
\end{prob}

\begin{sol}
	\begin{align*}
		\frac1{a+bi} &= \frac1{a+bi}\left(\frac{a-bi}{a-bi}\right) \\
			     &= \frac{a-bi}{a^2+b^2} \\
			     &= \frac a{a^2+b^2}-\frac b{a^2+b^2}i
	\end{align*}
	$$c=\frac a{a^2+b^2}\qquad d=-\frac b{a^2+b^2}$$
\end{sol}

\section{Question 5} %Problem n goes within this section

\begin{prob}
Problem 5 statement here.
\end{prob}

\begin{sol}
Solution to problem 5 solution.
\end{sol}


\end{document}
